\documentclass{article}
\usepackage[utf8]{inputenc}




\title{labwork 4}
\author{Pham Phan Bach }


\begin{document}

\maketitle
    

\section{Improvement explaination}
\begin{verbatim}
__global__ void grayscale2D(uchar3* input, uchar3* output, int width, 
int height){
	int tid_x = threadIdx.x + blockIdx.x * blockDim.x;
	int tid_y = threadIdx.y + blockIdx.y * blockDim.y;
	if(tid_x > width || tid_y > height) return;
	int tid = (int)(tid_x + tid_y * width);
	output[tid].x = (input[tid].x + input[tid].y + input[tid].z) / 3;
	output[tid].z = output[tid].y = output[tid].x;
}

void Labwork::labwork4_GPU() {
	int pixelCount = inputImage->width * inputImage->height;
	outputImage = static_cast<char*>(malloc(pixelCount * 3));
	uchar3* devImage;
    uchar3*	devOutput;
	cudaMalloc(&devImage, pixelCount * sizeof(uchar3));
	cudaMalloc(&devOutput, pixelCount * sizeof(uchar3));
	cudaMemcpy(devImage, inputImage->buffer, pixelCount * sizeof(uchar3),
	cudaMemcpyHostToDevice);


	int b_x = 32;
	int b_y = 32;
	int d_x = (int)(inputImage->width + b_x - 1) / b_x;
	int d_y = (int)(inputImage->height + b_y - 1) / b_y;
	
	dim3 blockSize = dim3(b_x, b_y);
	dim3 gridSize = dim3(d_x, d_y);
	grayscaleVer<<<gridSize, blockSize>>>(devImage, devOutput, inputImage->width, 
	inputImage->height);
	cudaMemcpy(outputImage, devOutput, pixelCount * sizeof(uchar3), 
	cudaMemcpyDeviceToHost);
	cudaFree(devImage);
	cudaFree(devOutput);
}


\end{verbatim}{}


\section{Result}
\begin{verbatim}
    USTH ICT Master 2018, Advanced Programming for HPC.
    Warming up...
    Starting labwork 4
    labwork 4 ellapsed 273.2ms
\end{verbatim}{}
\end{document}
